\documentclass{beamer}

\usetheme{CambridgeUS}
\title{Assignment 11} 
\author{Akhila Kumbha,CS21BTECH11031}
\date{\today}
\logo{\large \LaTeX{}}


\providecommand{\brak}[1]{\ensuremath{\left(#1\right)}}
\providecommand{\pr}[1]{\ensuremath{\Pr\left(#1\right)}}
\providecommand{\cbrak}[1]{\ensuremath{\left\{#1\right\}}}

\begin{document}


\begin{frame}
    \titlepage 
\end{frame}

\logo{}

\begin{frame}{Outline}
    \tableofcontents
\end{frame}


\section{Question}
\begin{frame}{Question}
Show that if\\ $\hat{E}\cbrak{s(t+\lambda)|s(t),s(t-\tau)}=\hat{E}\cbrak{s(t+\lambda)|s(t)}$ then $R_s(\tau)=Ie^{-\alpha|\tau|}$.\\
\end{frame}


\section{Solution}
\begin{frame}{Solution}
Since 
\begin{align}
    \hat{E}\cbrak{s(t+\lambda)|s(t)}=a\hspace{1mm}s(t)\\
     a=R(\lambda)/R(0)
\end{align}
it follows from the assumption that 
\begin{align}
    s(t+\lambda)-a\hspace{1mm}s(t)\perp s(t-\tau)
\end{align}
Hence
\begin{align}
    R(\lambda +\tau)=\frac{R(\lambda)}{R(0)} R(\tau)\label{eq1}
\end{align}
\end{frame} 

\begin{frame}
The only continuous function satisfying the above is an exponential.\\This is easily shown if we assume that $R(\lambda)$ is differentiable for $\lambda>0$.\\
Differentiating \eqref{eq1} with respect to $\lambda$ and setting $\lambda=0
^{+}$,we obtain 
\begin{align}
    R'(\tau)+\alpha R(\tau)=0\label{eq2}\\
    \alpha=\frac{-R'(0^+)}{R(0)},\tau>0
\end{align}
Equation \eqref{eq2} yields $R(\tau)=Ie^{-\alpha\tau}$ for $\tau>0$.

\end{frame}

\end{document}